\documentclass{article}
\usepackage[utf8]{inputenc}
\usepackage[spanish,mexico]{babel}
\usepackage[margin=2.5cm]{geometry}
\usepackage{graphicx}
\usepackage[export]{adjustbox}
\usepackage{caption}
\usepackage{subcaption}
\usepackage{fancyhdr}
\usepackage{avant}
\usepackage[T1]{fontenc}
\pagestyle{fancy}
\fancyhf{}
\renewcommand*\familydefault{\sfdefault}

\title{Proyecto Seguridad Informatica}
\author{Gonzalez Calzada Maximiliano}
\date{\today}

\begin{document}

    \thispagestyle{empty}
        
    \begin{figure}[ht]
        \minipage{0.16\textwidth}
            \includegraphics[width=2.2cm]{IPN_Logo.pdf}
            \label{LogoIPN}
        \endminipage
        \hspace{0.60\textwidth}
        \minipage{0.22\textwidth}
            \includegraphics[width=3cm]{UPIICSA_Logo.pdf}
            \label{LogoUPIICSA}
        \endminipage
    \end{figure}
	
	\begin{center}
        \vspace{0.8cm}
        \LARGE
        \textbf{INSTITUTO POLITÉCNICO NACIONAL}
        
        \vspace{0.8cm}
        \LARGE
        \textbf{UNIDAD PROFESIONAL INTERDISCIPLINARIA DE INGENIERÍA Y CIENCIAS SOCIALES Y ADMINISTRATIVAS}
        
        \vspace{1cm}	
        \Large
        \textbf{LICENCIATURA EN INGENIERÍA EN INFORMÁTICA}

        \vspace{1cm}
        \large	
        PRESENTAN \\
        \vspace{.3cm}
        \normalsize
        \begin{itemize}
            \item \textbf{Gonzalez Calzada Maximiliano}
            \item \textbf{Hernandez Garcia Juan Jose}
            \item \textbf{Maldonado Martínez Kevin Noel}
            \item \textbf{Martínez Lagunas Andrik Jeovany}
            \item \textbf{Muñoz Castro Angel Daniel}
        \end{itemize}
        
        \vspace{0.8cm}
        \normalsize	
        DOCENTE \\
        \vspace{.3cm}
        \large
        \textbf{Ortega Avalos Julia Alicia}
        
        \vspace{0.8cm}
        \normalsize	
        ASIGNATURA \\
        \vspace{.3cm}
        \large
        \textbf{Seguridad Informática}

        \vspace{0.8cm}
        \normalsize	
        SECUENCIA \\
        \vspace{.3cm}
        \large
        \textbf{4NM42}
        
        \vspace{0.8cm}
        \today
	\end{center}
	
	\newpage
	
    \section{Nombre del proyecto}

    \large
    "Web API Security Scanner"
    
    \section{Objetivo}

    Desarrollar una herramienta que escanee las API web en busca de vulnerabilidades de seguridad, asegurando que sean sólidas contra amenazas comunes y proporcionando recomendaciones de mejora.

    \subsection{Características Clave}

    \begin{itemize}
        \item \textbf{Detección Automatizada de Vulnerabilidad}: El escáner debe probar automáticamente las API en busca de vulnerabilidades comunes, incluyendo:
        \begin{itemize}
            \item \textbf{Defectos de autenticación}: Compruebe si hay mecanismos de autenticación débiles o faltantes (por ejemplo, JWT, OAuth).
            \item \textbf{Problemas de autorización}: Prueba de control de acceso incorrecto (por ejemplo, usuarios que acceden a los recursos que deberían).
            \item \textbf{Validación de entrada}: Identifique puntos finales vulnerables a la inyección SQL, secuencias de comandos en sitios cruzados (XSS) e inyección de comandos.
        \end{itemize}
        \item \textbf{Limitación de tarifas}: Asegúrese de que las API tengan una limitación de velocidad adecuada para prevenir el abuso (Ataques DDoS).
        \item \textbf{Exposición a Datos Sensibles}: Compruebe si la información confidencial (como contraseñas, números de tarjetas de crédito) se transmite sin cifrado.
        \item \textbf{Informes y Recomendaciones}: Después del escaneo, genere un informe detallado que resuma:
        \begin{itemize}
            \item Vulnerabilidades detectadas
            \item Niveles de riesgo (alto, medio, bajo)
            \item Pasos de remediación sugeridos o mejores prácticas.
        \end{itemize}
        \item \textbf{Interfaz amigable para el usuario}: Cree una interfaz de usuario simple que permita a los usuarios ingresar puntos finales y configuraciones de API fácilmente.
    \end{itemize}

    \subsection{Pasos de Implementación}

    \begin{enumerate}
        \item \textbf{Research Common API Vulnerabilities}: Familiarícese con OWASP API Security Top Ten y otras pautas de seguridad relevantes.
        \item \textbf{Elija un Lenguaje/Marco de Programación}: Seleccione un idioma (como Python o JavaScript) y un marco adecuado para construir el escáner.
        \item \textbf{Desarrolle Core Scanning Logic}: Implemente módulos para cada tipo de vulnerabilidad, aprovechando bibliotecas o herramientas cuando corresponda (por ejemplo, OWASP ZAP para integración).
        \item \textbf{Construir la interfaz de usuario}: Cree una interfaz web o de línea de comandos que permita a los usuarios ingresar sus detalles de API e iniciar escaneos.
        \item \textbf{Pruebas}: Pruebe el escáner contra API vulnerables conocidas para asegurarse de que detecta con precisión las vulnerabilidades.
        \item \textbf{Documentación}: Escriba documentación clara para los usuarios sobre cómo usar la herramienta e interpretar los resultados.
    \end{enumerate}
    
    \section{Alcance}

    \subsection{Cobertura de Seguridad:}
    \begin{itemize}
        \item \textbf{Autenticación y Autorización}: Revisar las configuraciones de autenticación (JWT, OAuth) y control de acceso.
        \item \textbf{Validación de Entrada}: Identificar vulnerabilidades de inyección (SQL, XSS, comandos).
        \item \textbf{Limitación de Tarifas y Exposición de Datos Sensibles}: Comprobar la limitación de tasa para evitar abusos y verificar si se transmiten datos sensibles sin cifrado.
        \item \textbf{Detección de Amenazas Comunes}: Enfocarse en las vulnerabilidades descritas en el OWASP API Security Top Ten.
    \end{itemize}

    \subsection{Límites del Proyecto:}
    \begin{itemize}
        \item \textbf{Exclusiones}: La herramienta no se enfocará en realizar pruebas de rendimiento o funcionalidad de la API; su único enfoque será la seguridad.
        \item \textbf{Limitaciones Técnicas}: El alcance inicial del proyecto no incluirá integración avanzada con sistemas de terceros para generación de reportes o monitoreo, aunque esta funcionalidad podría considerarse en versiones futuras.
    \end{itemize}

    \section{Justificación}

    En la actualidad, las API son fundamentales para el funcionamiento de aplicaciones modernas, permitiendo la comunicación entre diferentes sistemas y servicios. Con el aumento de su uso, también ha crecido la superficie de ataque, haciendo que la seguridad de las API sea una prioridad. A continuación se detallan las razones que apoyan la necesidad de un escáner de seguridad para APIs web:

    \begin{itemize}
        \item \textbf{Creciente Dependencia de APIs}: Con la propagación de arquitecturas basadas en microservicios y el uso extensivo de servicios web, las APIs se han convertido en un punto de entrada crucial para las aplicaciones. Esto las convierte en un objetivo atractivo para los atacantes.
        \item \textbf{Vulnerabilidades Comunes}: Existen vulnerabilidades comunes en las APIs, como las descritas en el OWASP API Security Top Ten. Un escáner automatizado puede ayudar a identificar y mitigar estas vulnerabilidades antes de que sean explotadas.
        \item \textbf{Automatización y Eficiencia}: Realizar análisis de seguridad manualmente es intensivo en tiempo y recursos. Una herramienta automatizada permite escanear múltiples APIs de forma eficiente, liberando a los equipos de seguridad para que se concentren en análisis más profundos y en la remediación de problemas detectados.
        \item \textbf{Mejora Continua en la Seguridad}: La generación de informes detallados con recomendaciones proporciona a los desarrolladores y equipos de seguridad una guía para mejorar continuamente la postura de seguridad de sus APIs.
        \item \textbf{Educación y Conciencia}: Incluir una interfaz amigable y documentación clara no solo hace que la herramienta sea accesible, sino que también educa a los usuarios sobre las prácticas de seguridad y la importancia de proteger sus APIs.
        \item \textbf{Cumplimiento Normativo}: Muchas organizaciones están sujetas a regulaciones que exigen la implementación de prácticas de seguridad adecuadas. Un escáner de seguridad puede ayudar a demostrar el cumplimiento de estas normativas.
        \item \textbf{Proactividad ante Amenazas}: La herramienta no solo identificará vulnerabilidades ya existentes, sino que también puede servir como un medio proactivo para detectar configuraciones inseguras y prácticas deficientes en la implementación de APIs.
    \end{itemize}

    \section{Marco teórico}

    \subsection{Introducción a las APIs Web}
    \textbf{Definición}: Las APIs (Interfaz de Programación de Aplicaciones) permiten que diferentes aplicaciones se comuniquen entre sí. Son cruciales en la arquitectura de software moderna, especialmente en aplicaciones web y móviles.

    \textbf{Importancia}: Facilitan la integración y la interoperabilidad, pero también presentan riesgos de seguridad que deben ser gestionados adecuadamente.

    \subsection{Amenazas Comunes a las APIs}
    \begin{itemize}
        \item \textbf{Inyección SQL}: Permite a los atacantes ejecutar consultas maliciosas en la base de datos.
        \item \textbf{XSS}: Aprovecha la confianza del navegador en las API para ejecutar código en el contexto de un usuario.
        \item \textbf{Falsificación de solicitudes entre sitios (CSRF)}: Permite a un atacante hacer solicitudes en nombre de un usuario autenticado.
        \item \textbf{Desbordamiento de datos}: Implica enviar un volumen alto de datos a la API con el objetivo de desestabilizar o hacerla inaccesible.
    \end{itemize}

    \section{Planeación}
    \section{Planteamiento de la solución}

\end{document}

